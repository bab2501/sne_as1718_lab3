%%%%%%%%%%%%%%%%%%%%%%%%%%%%%%%%%%%%%%%%%
% Beamer Presentation
% LaTeX Template
% Version 1.0 (10/11/12)
%
% This template has been downloaded from:
% http://www.LaTeXTemplates.com
%
% License:
% CC BY-NC-SA 3.0 (http://creativecommons.org/licenses/by-nc-sa/3.0/)
%
%%%%%%%%%%%%%%%%%%%%%%%%%%%%%%%%%%%%%%%%%

%----------------------------------------------------------------------------------------
%	PACKAGES AND THEMES
%----------------------------------------------------------------------------------------

\documentclass[10pt]{beamer}
\mode<presentation> {

% The Beamer class comes with a number of default slide themes
% which change the colors and layouts of slides. Below this is a list
% of all the themes, uncomment each in turn to see what they look like.

%\usetheme{default}
%\usetheme{AnnArbor}
%\usetheme{Antibes}
%\usetheme{Bergen}
%\usetheme{Berkeley}����JFIF����C
%\usetheme{Berlin}
%\usetheme{Boadilla}
%\usetheme{CambridgeUS}
%\usetheme{Copenhagen}
%\usetheme{Darmstadt}
%\usetheme{Dresden}
%\usetheme{Frankfurt}
%\usetheme{Goettingen}
%\usetheme{Hannover}
%\usetheme{Ilmenau}
%\usetheme{JuanLesPins}
%\usetheme{Luebeck}
%\usetheme{Madrid}
%\usetheme{Malmoe}
%\usetheme{Marburg}
%\usetheme{Montpellier}
\usetheme{PaloAlto}
%\usetheme{Pittsburgh}
%\usetheme{Rochester}
%\usetheme{Singapore}
%\usetheme{Szeged}
%\usetheme{Warsaw}

% As well as themes, the Beamer class has a number of color themes
% for any slide theme. Uncomment each of these in turn to see how it
% changes the colors of your current slide theme.

%\usecolortheme{albatross}
%\usecolortheme{beaver}
%\usecolortheme{beetle}
%\usecolortheme{crane}
%\usecolortheme{dolphin}
%\usecolortheme{dove}
%\usecolortheme{fly}
%\usecolortheme{lily}
%\usecolortheme{orchid}
%\usecolortheme{rose}
%\usecolortheme{seagull}
%\usecolortheme{seahorse}
\usecolortheme{whale}
%\usecolortheme{wolverine}

%\setbeamertemplate{footline} % To remove the footer line in all slides uncomment this line
\setbeamertemplate{footline}[page number] % To replace the footer line in all slides with a simple slide count uncomment this line

%\setbeamertemplate{navigation symbols}{} % To remove the navigation symbols from the bottom of all slides uncomment this line
}

\usepackage{graphicx} % Allows including images
\usepackage{booktabs} % Allows the use of \toprule, \midrule and \bottomrule in tables
\usepackage{amsmath}
\usepackage{subfig}
\usepackage{caption}
\usepackage{mathabx}
\usepackage{wasysym}
\usepackage{wrapfig}
\usepackage{tikz}
\usepackage{animate}
\usepackage{minted}
\usepackage{listings}
\usepackage{listings}
\usetikzlibrary{shapes.geometric, arrows}
\usepackage{minted}
\usepackage{color}

\usepackage{xcolor}
\usepackage{listings}
\lstset{basicstyle=\ttfamily,
  showstringspaces=false,
  commentstyle=\color{red},
  keywordstyle=\color{blue}
}

%\usepackage{subcaption}
%----------------------------------------------------------------------------------------
%	TITLE PAGE
%----------------------------------------------------------------------------------------
\title[]{Lab \#3: Software Defined Radio \\ ARDS signal analysis } % The short title appears at the bottom of every slide, the full title is only on the title page

\author{Alexander Blaauwgeers \\ \& \\ Bernardus Jansen} % Your name
\institute[University of Amsterdam] % Your institution as it will appear on the bottom of every slide, may be shorthand to save space
{
University of Amsterdam \\ % Your institution for the title page
\medskip
%\textit{john@smith.com} % Your email address
}
\date{April 20, 2018} % Date, can be changed to a custom date

\begin{document}

\begin{frame}
\titlepage % Print the title page as the first slide
\end{frame}

%----------------------------------------------------------------------------------
\section{Introduction }
\begin{frame}{Introduction}
\begin{block}{Question}
There are radio stations that send out RDS information so you know traffic jams are ahead or know the name of the
song that is currently playing.
\begin{itemize}
    \item Is there a way to retrieve this information using SDR?
    \item What kind of information can you get out of this?
    \item How does the protocol work?
\end{itemize}
\end{block}
\end{frame}

%----------------------------------------------------------------------------------
\section{Theory}
%----------------------------------------------------------------------------------
\begin{frame}{Theory}
\centering
\begin{itemize}
    \item 
\end{itemize}

\end{frame}

%----------------------------------------------------------------------------------
\section{Methodology}
\begin{frame}{Methodology}{}
We have developed and implemented following methodology:

\end{frame}

%----------------------------------------------------------------------------------
\section{Experiments}
\begin{frame}{Experiments}

\centering
\begin{figure}[h]
    %%\lstinputlisting[language=Python]{demo.py}
    %%\includegraphics[width=10cm]{code.png}%%

\end{figure}

\end{frame}

%----------------------------------------------------------------------------------
\section{Results}
\begin{frame}{Results}

\end{frame}


%----------------------------------------------------------------------------------

%----------------------------------------------------------------------------------




%\begin{frame}{Design}{Network physical topology}
%\includegraphics[width=\textwidth]{ANLab3Group3PhysicalDia.png}
%\end{frame}



\begin{frame}{Questions?}

Do you have any questions?

%\def\newblock{}
%\bibliographystyle{unsrt}
%\bibliography{mybib}
\end{frame}

\end{document} 
%---------------------------------------------------------------------------
